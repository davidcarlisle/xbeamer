% \iffalse meta-comment
%
% File: xbeamer-structure.dtx Copyright (C) 2024,2025 Joseph Wright
%
% It may be distributed and/or modified under the conditions of the
% LaTeX Project Public License (LPPL), either version 1.3c of this
% license or (at your option) any later version.  The latest version
% of this license is in the file
%
%    https://www.latex-project.org/lppl.txt
%
% This file is part of the "xbeamer bundle" (The Work in LPPL)
% and all files in that bundle must be distributed together.
%
% The released version of this bundle is available from CTAN.
%
% -----------------------------------------------------------------------
%
% The development version of the bundle can be found at
%
%    https://github.com/josephwright/xbeamer
%
% for those people who are interested.
%
% -----------------------------------------------------------------------
%
%<*driver>
\documentclass{l3doc}
% Additional commands needed in this source
\NewDocumentCommand\email{m}{\href{mailto:#1}{\nolinkurl{#1}}}
\begin{document}
  \DocInput{\jobname.dtx}
\end{document}
%</driver>
% \fi
%
% ^^A As we are dealing with a class, this has to be done manually
% \def\filedate{2025-03-19}
% \def\fileversion{v0.0.0}
%
% \title{^^A
%   \pkg{xbeamer-structure} -- Structural commands^^A
%   \thanks{This file describes \fileversion,
%     last revised \filedate.}^^A
% }
%
% \author{^^A
%  Joseph Wright^^A
%  \thanks{^^A
%    E-mail: \email{joseph@texdev.net}^^A
%   }^^A
% }
%
% \date{Released \filedate}
%
% \maketitle
%
% \begin{documentation}
%
% \end{documentation}
%
% \begin{implementation}
%
% \section{\pkg{xbeamer-structure} implementation}
%
% Start the \pkg{DocStrip} guards.
%    \begin{macrocode}
%<*class>
%    \end{macrocode}
%
% Identify the internal prefix.
%    \begin{macrocode}
%<@@=xbeamer>
%    \end{macrocode}
%
% \begin{macro}{\section, \subsection, \subsubsection}
%   At present stub commands: quite what needs to happen here is unclear.
%    \begin{macrocode}
\NewDocumentCommand \section { s O {#3} m } { }
\NewDocumentCommand \subsection { s O {#3} m } { }
\NewDocumentCommand \subsubsection { s O {#3} m } { }
%    \end{macrocode}
% \end{macro}
%
% \begin{macro}{\frametitle}
%   As a interim approach, make frame titles section commands: this will allow
%   tagging but does leave the question of how to handle this longer-term.
%   The values are those from \cls{article} at the moment: this will need to
%   be templated later.
%    \begin{macrocode}
\NewDocumentCommand \frametitle { D <> { all } O {#3} m }
  {
    \@startsection
      { section }
      { 1 }
      { 0pt }
      { -3.5ex plus -1ex minus -0.2ex }
      {  2.3ex plus 0.2ex }
      { \normalfont \Large \bfseries }
      {#3}
  }
%    \end{macrocode}
% \end{macro}
%
% Temporary code needed to allow frame titles to look like normal sections:
% this will all need to go later. In particular, \cls{beamer} does not
% define \cs{thepage} as it does not have pages; the lower-level \cs{c@page}
% is used instead.
%    \begin{macrocode}
\setcounter { secnumdepth } { -1 }
\newcommand \thepage { \csname @arabic\endcsname \c@page }
%    \end{macrocode}
%
% \begin{environment}{description, quote, quotation, verse}
%   Stub logical environments: needed as the tagging setup expects these
%   to exist.
%    \begin{macrocode}
\NewDocumentEnvironment { description } { } { } { }
\NewDocumentEnvironment { quote } { } { } { }
\NewDocumentEnvironment { quotation } { } { } { }
\NewDocumentEnvironment { verse } { } { } { }
%    \end{macrocode}
% \end{environment}
%
% \begin{macro}{\block}
%    \begin{macrocode}
\NewDocumentEnvironment { block } { D <> { all } }
  {
    \vbox_set:Nw \l_@@_tmp_box
  }
  {
    \vbox_set_end:
    \box_use:N \l_@@_tmp_box
  }
%    \end{macrocode}
% \end{macro}
%
% \begin{macro}{\blocktitle}
%   As a interim approach, make block titles subsection commands.
%   The values are those from \cls{article} at the moment: this definitely needs
%   to be completely re-worked! Like \cs{frametitle}, we presumably want this
%   command to save the data, which is then typeset by a suitable template.
%   The latter will likely need to be triggered in the end-of-block code
%   as only then is the title available.
%    \begin{macrocode}
\NewDocumentCommand \blocktitle { D <> { all } m }
  {
    \@startsection
      { subsection }
      { 2 }
      { 0pt }
      { -3.25ex plus -1ex minus -0.2ex }
      {  1.5ex plus 0.2ex }
      { \normalfont \large \bfseries }
      {#2}
  }
%    \end{macrocode}
% \end{macro}
%
% \begin{macro}{\alert}
%   We will likely want more flexibility here: for the moment, hard-code
%   something.
%    \begin{macrocode}
\NewDocumentCommand \alert { D <> { all } +m }
  {
    \@@_overlay:nTF {#1}
      { \textcolor { red } }
      { \textcolor { . } }
        {#2}
  }
%    \end{macrocode}
% \end{macro}
%
% \begin{macro}{\item}
% \begin{macro}{\@@_item_parse_spec:n}
% \begin{macro}{\__kernel_list_item_end:}
%   Again, add the additional argument: here, we have to do a little gymnastics.
%    \begin{macrocode}
\AddToHook { begindocument / before }
  {
    \NewCommandCopy \stditem \item
    \RenewDocumentCommand \item { d <> = { label } o }
      {
        \IfNoValueTF {#1}
          {
            \tl_if_empty:NF \l_@@_block_inner_spec_tl
              {
                \exp_args:NV \@@_item_parse_spec:n
                  \l_@@_block_inner_spec_tl
              }
          }
          { \@@_item_parse_spec:n {#1} }
        \IfNoValueTF {#2}
          { \stditem }
          { \stditem [ #2 ] }
      }
  }
%    \end{macrocode}
%   Parsing the spec is a separate function here has there are a couple of
%   routes to get here. At present we only have a \texttt{false} branch, but
%   for spacing we likely will need to add something to the \texttt{true}
%   branch too.
%    \begin{macrocode}
\cs_new_protected:Npn \@@_item_parse_spec:n #1
  {
    \@@_overlay:nF {#1}
      {
        \tl_gset:Nn \g_@@_list_end_tl
          { \draw_hidden_end: }
        \draw_hidden_begin:
      }
  }
%    \end{macrocode}
%   No hook at present so we have to hard-code a redefinition.
%    \begin{macrocode}
\cs_gset_protected:Npe \__kernel_list_item_end:
  {
    \exp_not:N \g_@@_list_end_tl
    \tl_gclear:N \exp_not:N \g_@@_list_end_tl
    \exp_not:o { \__kernel_list_item_end: }
  }
%    \end{macrocode}
% \end{macro}
% \end{macro}
% \end{macro}
% \begin{variable}{\tl_new:N \g_@@_list_end_tl}
%    \begin{macrocode}
\tl_new:N \g_@@_list_end_tl
%    \end{macrocode}
% \end{variable}
%
% \begin{variable}{\l_@@_block_inner_spec_tl}
%   Add an overlay key to the \texttt{block} template.  Placed here, it applies
%   \emph{before} the \cs{item} starts, so we do not have to redefine the
%   latter to do actions up-front. This also means it can apply to whatever we
%   want it to within a block.
%    \begin{macrocode}
\keys_define:nn { template / block / display  }
  { overlay-spec .tl_set:N = \l_@@_block_inner_spec_tl }
%    \end{macrocode}
% \end{variable}
%
% Again temporary code: allow us to have some useful result for lists, and in
% particular let them interface with tagging (without this, you cannot load the
% tagging code).
%    \begin{macrocode}
\setlength  \labelsep  { 0.5em }
\newcommand \labelenumi { \theenumi . }
\newcommand \labelenumii { ( \theenumii ) }
\newcommand \labelenumiii { \theenumiii . }
\newcommand \labelenumiv { \theenumiv . }
\newcommand \labelitemi  { \labelitemfont \textbullet }
\newcommand \labelitemii { \labelitemfont \bfseries \textendash }
\newcommand \labelitemiii { \labelitemfont \textasteriskcentered }
\newcommand \labelitemiv  { \labelitemfont \textperiodcentered }
\newcommand \labelitemfont { \normalfont }
%    \end{macrocode}
%
% List design settings: once again temporary and taken from the standard
% classes.
%    \begin{macrocode}
\setlength \leftmargini   { 2em }
\setlength \leftmarginii  { 2em }
\setlength \leftmarginiii { 2em }
\setlength  \labelsep  { 0.5em }
\setlength  \labelwidth { \leftmargini }
\addtolength \labelwidth { -\labelsep }
\def \@listi
  {
    \leftmargin \leftmargini
    \topsep 3pt plus 2pt minus 2.5pt
    \parsep 0pt
    \itemsep3pt plus 2pt minus 3pt
  }
\let \@listI \@listi
\def \@listii
  {
    \leftmargin\leftmarginii
    \topsep    2pt plus 1pt minus 2pt
    \parsep    0pt plus pt
    \itemsep   \parsep
  }
\def \@listiii 
  {
    \leftmargin\leftmarginiii
    \topsep    2pt plus 1pt minus 2pt
    \parsep    0pt plus pt
    \itemsep   \parsep
  }
\setlength \partopsep { 0pt }
%    \end{macrocode}
%
%    \begin{macrocode}
%</class>
%    \end{macrocode}
%
% \end{implementation}
%
% \PrintIndex
