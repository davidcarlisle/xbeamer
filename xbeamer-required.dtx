% \iffalse meta-comment
%
% File: xbeamer-required.dtx Copyright (C) 2025 Joseph Wright
%
% It may be distributed and/or modified under the conditions of the
% LaTeX Project Public License (LPPL), either version 1.3c of this
% license or (at your option) any later version.  The latest version
% of this license is in the file
%
%    https://www.latex-project.org/lppl.txt
%
% This file is part of the "xbeamer bundle" (The Work in LPPL)
% and all files in that bundle must be distributed together.
%
% The released version of this bundle is available from CTAN.
%
% -----------------------------------------------------------------------
%
% The development version of the bundle can be found at
%
%    https://github.com/josephwright/xbeamer
%
% for those people who are interested.
%
% -----------------------------------------------------------------------
%
%<*driver>
\documentclass{l3doc}
% Additional commands needed in this source
\NewDocumentCommand\email{m}{\href{mailto:#1}{\nolinkurl{#1}}}
\begin{document}
  \DocInput{\jobname.dtx}
\end{document}
%</driver>
% \fi
%
% ^^A As we are dealing with a class, this has to be done manually
% \def\filedate{2025-03-19}
% \def\fileversion{v0.0.0}
%
% \title{^^A
%   \pkg{xbeamer-required} -- \enquote{Required} definitions^^A
%   \thanks{This file describes \fileversion,
%     last revised \filedate.}^^A
% }
%
% \author{^^A
%  Joseph Wright^^A
%  \thanks{^^A
%    E-mail: \email{joseph@texdev.net}^^A
%   }^^A
% }
%
% \date{Released \filedate}
%
% \maketitle
%
% \begin{documentation}
%
% \end{documentation}
%
% \begin{implementation}
%
% \section{\pkg{xbeamer-required} implementation}
%
% Start the \pkg{DocStrip} guards.
%    \begin{macrocode}
%<*class>
%    \end{macrocode}
%
% Identify the internal prefix.
%    \begin{macrocode}
%<@@=xbeamer>
%    \end{macrocode}
%
% Here we collect up things that are more-or-less required to create a useful
% class but are not defined by the \LaTeX{} kernel for historical reasons. They
% are therefore largely copies from \texttt{article.cls} and contain
% \enquote{classical} definitions so that they follow the expectations of
% third-party code.
%
% \begin{macro}{\today}
%   This is the definition as done in the standard classes.
%    \begin{macrocode}
\cs_new_nopar:Npn \today
  {
    \ifcase \month \or
      January \or
      February \or
      March \or
      April \or
      May \or
      June \or
      July \or
      August \or
      September \or
      October \or
      November \or
      December
    \fi
    \space
    \number \day ,
    \number \year
  }
%    \end{macrocode}
% \end{macro}
%
% \subsection{Standard design settings}
%
%    \begin{macrocode}
\setcounter { tocdepth } { 3 }
\setlength \arraycolsep { 5pt }
\setlength \tabcolsep { 6pt }
\setlength \arrayrulewidth { 0.4pt }
\setlength \doublerulesep { 2pt }
\setlength \tabbingsep { \labelsep }
\skip \@mpfootins = \skip \footins
\setlength \fboxsep { 3pt }
\setlength \fboxrule { 0.4pt }
%    \end{macrocode}
%
% \subsection{List support}
%
%    \begin{macrocode}
\setlength  \labelsep  { 0.5em }
\cs_new:Npn \labelenumi { \theenumi . }
\cs_new:Npn \labelenumii { ( \theenumii ) }
\cs_new:Npn \labelenumiii { \theenumiii . }
\cs_new:Npn \labelenumiv { \theenumiv . }
\cs_new:Npn \labelitemi  { \labelitemfont \textbullet }
\cs_new:Npn \labelitemii { \labelitemfont \bfseries \textendash }
\cs_new:Npn \labelitemiii { \labelitemfont \textasteriskcentered }
\cs_new:Npn \labelitemiv  { \labelitemfont \textperiodcentered }
\cs_new:Npn \labelitemfont { \normalfont }
%    \end{macrocode}
%
%    \begin{macrocode}
\setlength \leftmargini   { 2em }
\setlength \leftmarginii  { 2em }
\setlength \leftmarginiii { 2em }
\setlength  \labelsep  { 0.5em }
\setlength  \labelwidth { \leftmargini }
\addtolength \labelwidth { -\labelsep }
\cs_gset_nopar:Npn \@listi
  {
    \leftmargin \leftmargini
    \topsep 3pt plus 2pt minus 2.5pt
    \parsep 0pt
    \itemsep 3pt plus 2pt minus 3pt
  }
\cs_gset_eq:NN \@listI \@listi
\cs_gset_nopar:Npn \@listii
  {
    \leftmargin \leftmarginii
    \topsep    2pt plus 1pt minus 2pt
    \parsep    0pt plus 1pt
    \itemsep   \parsep
  }
\cs_gset_nopar:Npn \@listiii
  {
    \leftmargin \leftmarginiii
    \topsep    2pt plus 1pt minus 2pt
    \parsep    0pt plus 1pt
    \itemsep   \parsep
  }
\setlength \partopsep { 0pt }
%    \end{macrocode}
%
%    \begin{macrocode}
%</class>
%    \end{macrocode}
%
% \end{implementation}
%
% \PrintIndex
