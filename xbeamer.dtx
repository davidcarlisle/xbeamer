% \iffalse meta-comment
%
% File: xbeamer.dtx Copyright (C) 2023-2025 Joseph Wright
%
% It may be distributed and/or modified under the conditions of the
% LaTeX Project Public License (LPPL), either version 1.3c of this
% license or (at your option) any later version.  The latest version
% of this license is in the file
%
%    https://www.latex-project.org/lppl.txt
%
% This file is part of the "xbeamer bundle" (The Work in LPPL)
% and all files in that bundle must be distributed together.
%
% The released version of this bundle is available from CTAN.
%
% -----------------------------------------------------------------------
%
% The development version of the bundle can be found at
%
%    https://github.com/josephwright/xbeamer
%
% for those people who are interested.
%
% -----------------------------------------------------------------------
%
%<*driver>
\documentclass{l3doc}
% Additional commands needed in this source
\NewDocumentCommand\email{m}{\href{mailto:#1}{\nolinkurl{#1}}}
\begin{document}
  \DocInput{\jobname.dtx}
\end{document}
%</driver>
% \fi
%
% ^^A As we are dealing with a class, this has to be done manually
% \def\filedate{2025-03-19}
% \def\fileversion{v0.0.0}
%
% \title{^^A
%   \pkg{xbeamer} -- Overall set up^^A
%   \thanks{This file describes \fileversion,
%     last revised \filedate.}^^A
% }
%
% \author{^^A
%  Joseph Wright^^A
%  \thanks{^^A
%    E-mail: \email{joseph@texdev.net}^^A
%   }^^A
% }
%
% \date{Released \filedate}
%
% \maketitle
%
% \begin{documentation}
%
% \end{documentation}
%
% \begin{implementation}
%
% \section{\pkg{xbeamer} implementation}
%
% Start the \pkg{DocStrip} guards.
%    \begin{macrocode}
%<*class>
%    \end{macrocode}
%
% Identify the internal prefix.
%    \begin{macrocode}
%<@@=xbeamer>
%    \end{macrocode}
%
% \subsection{Set up}
%
% Identify the package and give the over all version information.
%    \begin{macrocode}
\ProvidesExplClass {xbeamer} {2025-03-19} {0.0.0}
  {A class for typesetting presentations}
%    \end{macrocode}
%
% \subsection{Extra variants}
%
%    \begin{macrocode}
\cs_generate_variant:Nn \hook_gput_code:nnn { nne }
%    \end{macrocode}
%
% \subsection{Scratch space}
%
% \begin{macro}{\@@_tmp:w}
%   For one-off processing.
%    \begin{macrocode}
\cs_new_protected:Npn \@@_tmp:w { }
%    \end{macrocode}
% \end{macro}
%
% \begin{variable}{\l_@@_tmp_box}
%    \begin{macrocode}
\box_new:N \l_@@_tmp_box
%    \end{macrocode}
% \end{variable}
%
% \subsection{Option handling}
%
% \begin{variable}
%   {
%     \l_@@_aspect_ratio_str ,
%     \l_@@_fontsize_dim     ,
%     \l_@@_mode_str
%   }
%    \begin{macrocode}
\keys_define:nn { xbeamer }
  {
    aspect-ratio .str_set:N =
      \l_@@_aspect_ratio_str ,
    font-size .dim_set:N =
      \l_@@_fontsize_dim ,
    mode .choices:nn =
      { handout , projector }
      { \str_set:NV \l_@@_mode_str \l_keys_choice_tl } ,
    unknown .code:n =
      {
        \msg_error:nne { xbeamer } { unknown-class-option }
          { \l_keys_key_str \tl_if_empty:nF {#1} { = \tl_to_str:n {#1} } }
      }
  }
%    \end{macrocode}
% \end{variable}
%
% Scope for options.
%    \begin{macrocode}
\keys_define:nn { xbeamer }
  {
    aspect-ratio .usage:n = load ,
    font-size    .usage:n = load ,
    mode         .usage:n = load
  }
%    \end{macrocode}
%
% Initial values
%    \begin{macrocode}
\keys_set:nn { xbeamer }
  {
    aspect-ratio = 16:9 ,
    font-size    = 11pt ,
    mode         = projector
  }
%    \end{macrocode}
%
%    \begin{macrocode}
\msg_new:nnnn { xbeamer } { unknown-class-option }
  { Unknown~class~option~"#1". }
  {
    The~xbeamer~class~received~the~key-value~option~"#1",~
    but~this~is~not~recognised.
  }
%    \end{macrocode}
%
%    \begin{macrocode}
\ProcessKeyOptions
%    \end{macrocode}
%
% \subsection{Setting up}
%
%  Load the font size setup if available, otherwise fall back on scaling.
%    \begin{macrocode}
\file_if_exist_input:nF { size \dim_to_decimal:n \l_@@_fontsize_dim .clo }
  {
    \file_input:n { size10.clo }
    \RequirePackage { relsize }
    \hook_gput_code:nne { begindocument } { xbeamer }
      { \exp_not:N \relsize { \fp_eval:n { \l_@@_fontsize_dim / 10pt } } }
  }
%    \end{macrocode}
%
% \begin{variable}{\c_@@_paper_height_dim, \c_@@_paper_width_dim}
%   As \pkg{geometry} is being used to set the paper size with no previous
%   value, we have to use the optional argument rather than waiting to
%   apply \cs{geometry}.
%    \begin{macrocode}
\dim_const:Nn \c_@@_paper_height_dim { 96mm }
\use:e
  {
    \cs_set_protected:Npn \exp_not:N \@@_tmp:w
      #1 \tl_to_str:n { : } #2 \tl_to_str:n { : } #3 \exp_not:N \q_stop
      {
        \dim_const:Nn \exp_not:N \c_@@_paper_width_dim
          {
            \exp_not:N \fp_to_dim:n
              { (#1 / #2) * \exp_not:N \c_@@_paper_height_dim }
          }
      }
    \exp_not:N \@@_tmp:w \l_@@_aspect_ratio_str
      \tl_to_str:n { : } 100 \exp_not:N \q_stop
  }
\use:e
  {
    \exp_not:N \RequirePackage
      [
        papersize =
          {
            \dim_use:N \c_@@_paper_width_dim  ,
            \dim_use:N \c_@@_paper_height_dim
          } ,
        hmargin = 10mm ,
        top     = 6mm ,
        bottom  = 6mm  ,
        headsep = 0mm
      ]
      { geometry }
  }
%    \end{macrocode}
% \end{variable}
%
% Turn off justification
%    \begin{macrocode}
\raggedright
%    \end{macrocode}
%
% \subsection{Standard design settings}
%
% These are inherited from the \pkg{article}: in \pkg{beamer} they are in a
% separate file, but they make sense here.
%
%    \begin{macrocode}
\setcounter { tocdepth } { 3 }
\setlength \arraycolsep { 5pt }
\setlength \tabcolsep { 6pt }
\setlength \arrayrulewidth { 0.4pt }
\setlength \doublerulesep { 2pt }
\setlength \tabbingsep { \labelsep }
\skip\@mpfootins = \skip\footins
\setlength \fboxsep { 3pt }
\setlength \fboxrule { 0.4pt }
%    \end{macrocode}
%
% \subsection{Font selection}
%
% The aim here is to minimize change from the standard font setup but at the
% same time provide a sanserif default. Since \pkg{beamer} was released,
% better sanserif math mode fonts have become available. For OpenType engines,
% requiring \pkg{unicode-math} is the most sensible approach. The New Computer
% Modern font provides a reasonable initial set of glyphs. It comes with a
% wrapper package, but that does various other things: if the user wants these,
% they can choose to load themselves. For 8-bit engines, switching the text
% font to be sanserif is easy. For math mode, the \pkg{sansmathfonts} package
% does a good job: here, using the package rather than  adjusting directly is
% the sensible option.
%    \begin{macrocode}
\sys_if_engine_opentype:TF
  {
    \RequirePackage { unicode-math }
    \setmainfont { NewCMSans10-Regular.otf }
    \setmathfont { NewCMSansMath-Regular.otf }
  }
  {
    \RequirePackage { sansmathfonts }
    \cs_set_eq:NN \rmdefault \sfdefault
  }
%    \end{macrocode}
%
% \subsection{Hyperlinks}
%
% A requirement.
%    \begin{macrocode}
\RequirePackage { hyperref }
%    \end{macrocode}
%
% \subsection{Tagging}
%
% We need to extend the standard tagging model to work with slides and so on.
%    \begin{macrocode}
\tagpdfsetup
  {
    role / new-tag = frame / Sect         ,
    role / new-tag = frametitle / Caption ,
    role / new-tag = slide / Div
  } 
%    \end{macrocode}
%
%    \begin{macrocode}
%</class>
%    \end{macrocode}
%
% \end{implementation}
%
% \PrintIndex
