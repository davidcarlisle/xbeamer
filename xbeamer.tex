\iffalse meta-comment

File: xbeamer.tex Copyright (C) 2023-2025 Joseph Wright

It may be distributed and/or modified under the conditions of the
LaTeX Project Public License (LPPL), either version 1.3c of this
license or (at your option) any later version.  The latest version
of this license is in the file

   https://www.latex-project.org/lppl.txt

This file is part of the "xbeamer bundle" (The Work in LPPL)
and all files in that bundle must be distributed together.

The released version of this bundle is available from CTAN.

-----------------------------------------------------------------------

The development version of the bundle can be found at

   https://github.com/josephwright/xbeamer

for those people who are interested.

-----------------------------------------------------------------------

\fi

\DocumentMetadata{pdfversion = 2.0, lang = en}

\documentclass{l3doc}

\usepackage{siunitx}

% Commands needed in this document
\ExplSyntaxOn
\makeatletter
\NewDocumentCommand \acro { m }
  {
    \textsc
      {
        \exp_args:NV \tl_if_head_eq_charcode:nNTF \f@series { m }
          { \text_lowercase:n }
          { \use:n }
            {#1}
      }
  }
\makeatother
\ExplSyntaxOff
\NewDocumentCommand\foreign{m}{\textit{#1}}
\NewDocumentCommand\email{m}{\href{mailto:#1}{\nolinkurl{#1}}}
% Tidy up the above in bookmarks
\makeatletter
\pdfstringdefDisableCommands{%
  \let\acro\@firstofone
  \let\foreign\@firstofone
}
\makeatother

% As we are dealing with a class, this has to be done manually
\def\filedate{2025-03-19}
\def\fileversion{v0.0.0}

\begin{document}

\title{%
  \pkg{xbeamer} -- A class for typesetting presentations%
  \thanks{This file describes \fileversion,
    last revised \filedate.}%
}

\author{%
  Joseph Wright%
  \thanks{%
    E-mail: \email{joseph@texdev.net}%
  }%
}

\date{Released \filedate}

\maketitle

\tableofcontents

\begin{documentation}

\section{Introduction%
  \label{sec:intro}}

The \pkg{beamer} class was first released in 2003, and rapidly became the
most populat method for producing presentations in \LaTeX{}. As detailed in
the \pkg{beamer} manual
\begin{quotation}
  Till Tantau created \pkg{beamer} mainly in his spare time. Many other people
  have helped by sending him emails containing suggestions for improvement or
  corrections or patches or whole new themes (by now, this amounts to over a
  thousand emails concerning \pkg{beamer}). Indeed, most of the development was
  only initiated by feature requests and bug reports. Without this feedback,
  \pkg{beamer} would still be what it was originally intended to be: a small
  private collection of macros that make using the \pkg{seminar} class easier.
  Till created the first version of \pkg{beamer} for his PhD defense
  presentation in February 2003. A month later, he put the package on
  \acro{ctan} at the request of some colleagues. After that, things somehow got
  out of hand.
\end{quotation}

The effort which Till put in over the time he was developing \pkg{beamer}
cannot be underestimated. However, there are several challenges which confront
us today
\begin{enumerate}
  \item The document interface is flexible but in places deviates from normal
    \LaTeX{} conventions
  \item Internally, the code shows that Till was learning \LaTeX{} programming
    as he wrote \pkg{beamer}, and was coding whatever was needed to get the
    visual results
  \item Till made few comments in the code or in commit messages in the code
    history
  \item The underlying box structure of a \pkg{beamer} document is very
    different from the standard \LaTeX{} model, and a lot of material is
    boxed up multiple times
\end{enumerate}
These all feed into an issue addressing a major requirement today:
accessibility. Broadly, the internal structure (and to some extend the user
interface) of \pkg{beamer} mean that it is not possible to \enquote{retrofit}
PDF tagging into the class.

Instead, the approach is to develop a new class, currently called
\pkg{xbeamer}, which takes ideas from \pkg{beamer} but with tagging and
accessibility of structure as a design aim from the beginning. In contrast to
work by the \LaTeX{} Project Team on making the core classes accessible, the
expectation for \pkg{xbeamer} is that users \emph{will} need to change their
sources. Unlike other documents, presentations tend to be \enquote{single use}:
revised and adjusted each time they are used. The need to look edit sources
should therefore not be an insurmountable barrier.

At present, this code is \emph{highly} experimental: only a (small) subset of
\pkg{beamer} functionality is implemented, some things are being done
differently, almost everything is still subject to discussion.

\section{Simple example documents}

Currently, \pkg{xbeamer} \emph{absolutely requires} the use of the
\cs{DocumentMetadata} command \emph{and} the \texttt{testphase} code for new
document structures. As such, the most basic \pkg{xbeamer} document is
\begin{verbatim}
  \DocumentMetadata{testphase = latest}
  \documentclass{xbeamer}
  \begin{document}
  \begin{frame}
    Some content here
  \end{frame}
  \end{document}
\end{verbatim}

A slightly more useful version, which generates multiple slides and shows some
(current) features, is
\begin{verbatim}
  \DocumentMetadata{testphase = latest}
  \documentclass{xbeamer}
  \begin{document}
  \begin{frame}
    \frametitle{An example frame}
    \begin{itemize}[overlay-spec = +-]
      \item This will be on slide one onward
      \item This will be on slide two onward
      \item<.-> So will this
      \item But this will only appear on slide three
    \end{itemize}
    Back to appearing on all slides
  \end{frame}
  \end{document}
\end{verbatim}

Tagging is activated for the standard (projector) output of \pkg{xbeamer}, but
it is more useful in handout mode, which is activated using a class option.
\begin{verbatim}
  \DocumentMetadata{testphase = latest}
  \documentclass[mode = handout]{xbeamer}
  \begin{document}
  \begin{frame}
    \frametitle{An example frame}
    \begin{itemize}[overlay-spec = +-]
      \item This will be on slide one onward
      \item This will be on slide two onward
      \item<.-> So will this
      \item But this will only appear on slide three
    \end{itemize}
    Back to appearing on all slides
  \end{frame}
  \end{document}
\end{verbatim}

\section{Class structure and design decisions}

As covered in Section~\ref{sec:intro}, the \pkg{xbeamer} class is currently
highly experimental. Active discussion is ongoing around many aspects of the
way that things should work, and very little is therefore at all stable. That
said, some decisions have been made: some of this is re-stating ideas which
carry forward from \pkg{beamer}.
\begin{itemize}
  \item The basic structure of a presentation is made up of \texttt{frame}
    environments, which are made up of one or more slides.
  \item Variable content is indicated by an \emph{overlay specification}, given
    in optional angle brackets (|< ... >|).
  \item Slides have a \emph{fixed} height (the height itself is not yet
    decided, but is likely to be either \qty{96}{\mm} or \qty{100}{\mm}).
  \item The default font will be sanserif using established standard
    implementations (currently \pkg{sansmathfonts} for \pdfTeX{} and New
    Computer Modern for OpenType engines).
\end{itemize}

\end{documentation}

\section{Creating frames}

\subsection{The frame environment}

\DescribeEnv{frame}
A presentation consists of a series of frames. Each frame consists of a series
of slides. You create a frame using |frame| environment. All of the text that
is not tagged by overlay specifications is shown on all slides of the frame.
(Overlay specifications are explained in more detail in later sections. For the
moment, let's just say that an overlay specification is a list of numbers or
number ranges in pointed brackets that is put after certain commands as in
|\uncover<1,2>{Text}|.) If a frame contains commands that have an overlay
specification, the frame will contain multiple slides; otherwise it contains
only one slide.

\begin{verbatim}
  \begin{frame}
    \frametitle{A title}
    Some content
  \end{frame}
\end{verbatim}

\subsection{Frame and margin size}

\subsection{Restricting the slides of a frame}

The number of slides in a frame is automatically calculated. If the largest
number mentioned in any overlay specification inside the frame is 4, four
slides are introduced (despite the fact that a specification like |<4->| might
suggest that more than four slides would be possible).

\section{Creating overlays}

\subsection{The general concept of overlay specifications}

\subsection{Commands with overlay specifications}

\subsection{Environments with overlay specifications}

\subsection{Dynamically changing text or images}

\section{Structuring a presentation: the local structure}

\LaTeX{} provides different commands for structuring text \enquote{locally},
for example, \foreign{via} the |itemize| environment. These environments are
also available in the \pkg{xbeamer} class, although their appearance has been
slightly changed.

\subsection{Itemizations, enumerations and descriptions}

\subsection{Highlighting}

\subsection{Block environments}

\subsection{Splitting a frame into multiple columns}

\PrintIndex

\end{document}
