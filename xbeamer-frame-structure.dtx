% \iffalse meta-comment
%
% File: xbeamer-frame-structure.dtx Copyright (C) 2025 Joseph Wright
%
% It may be distributed and/or modified under the conditions of the
% LaTeX Project Public License (LPPL), either version 1.3c of this
% license or (at your option) any later version.  The latest version
% of this license is in the file
%
%    https://www.latex-project.org/lppl.txt
%
% This file is part of the "xbeamer bundle" (The Work in LPPL)
% and all files in that bundle must be distributed together.
%
% The released version of this bundle is available from CTAN.
%
% -----------------------------------------------------------------------
%
% The development version of the bundle can be found at
%
%    https://github.com/josephwright/xbeamer
%
% for those people who are interested.
%
% -----------------------------------------------------------------------
%
%<*driver>
\documentclass{l3doc}
% Additional commands needed in this source
\NewDocumentCommand\email{m}{\href{mailto:#1}{\nolinkurl{#1}}}
\begin{document}
  \DocInput{\jobname.dtx}
\end{document}
%</driver>
% \fi
%
% ^^A As we are dealing with a class, this has to be done manually
% \def\filedate{2025-03-19}
% \def\fileversion{v0.0.0}
%
% \title{^^A
%   \pkg{xbeamer-frame} -- The structure of frames^^A
%   \thanks{This file describes \fileversion,
%     last revised \filedate.}^^A
% }
%
% \author{^^A
%  Joseph Wright^^A
%  \thanks{^^A
%    E-mail: \email{joseph@texdev.net}^^A
%   }^^A
% }
%
% \date{Released \filedate}
%
% \maketitle
%
% \begin{documentation}
%
% \end{documentation}
%
% \begin{implementation}
%
% \section{\pkg{xbeamer-frame-structure} implementation}
%
% Start the \pkg{DocStrip} guards.
%    \begin{macrocode}
%<*class>
%    \end{macrocode}
%
% Identify the internal prefix.
%    \begin{macrocode}
%<@@=xbeamer>
%    \end{macrocode}
%
% \begin{environment}{columns}
%   We leave vertical mode here as otherwise there is an issue with tagging
%   due to box nesting: see
%   \url{https://github.com/latex3/tagging-project/issues/54}. That contrasts
%   with \pkg{beamer}: may need to revisit later.
%    \begin{macrocode}
\NewDocumentEnvironment { columns } { D <> { all } }
  {
    \mode_leave_vertical:
    \hbox_set_to_wd:Nnw \l_@@_tmp_box { \textwidth }
      \ignorespaces
  }
  {
      \unskip
    \hbox_set_end:
    \box_use:N \l_@@_tmp_box
  }
%    \end{macrocode}
% \end{environment}
%
% \begin{variable}{\l_@@_column_alignment_tl}
%    \begin{macrocode}
\keys_define:nn { xbeamer / column }
  {
    vertical-alignment .choices:nn =
      { bottom , center , stretch , top }
      {
        \tl_set_eq:NN \l_@@_column_alignment_tl
          \l_keys_value_tl
      }
  }
\keys_set:nn { xbeamer / column }
  {
    vertical-alignment = center
  }
%    \end{macrocode}
% \end{variable}
%
% \begin{environment}{column}
%    \begin{macrocode}
\NewDocumentEnvironment { column } { D <> { all } O { } m }
  {
    \keys_set:nn { xbeamer / column } {#2}
    \use:e
      {
        \exp_not:N \begin { minipage }
          [ \tl_head:V \l_@@_column_alignment_tl ]
      }
        {#3}
      \leavevmode
      \raggedright
      \ignorespaces
  }
%    \end{macrocode}
%   The \cs{@ignore} here means that any spaces after |\end{column}| are
%   suppressed by a \tn{ignorespaces} inserted by the kernel.
%    \begin{macrocode}
  {
    \end { minipage }
    \@ignoretrue
  }
%    \end{macrocode}
% \end{environment}
%
%    \begin{macrocode}
%</class>
%    \end{macrocode}
%
% \end{implementation}
%
% \PrintIndex
